\documentclass[fleqn,final]{beamer}
\mode<presentation>
{  \usetheme{I6dv} }
\usepackage{times}
\usepackage{etex}
\usepackage{amsmath,amssymb}
%\usepackage{sfmath} % for sans serif math fonts; wget http://dtrx.de/od/tex/sfmath.sty
\usepackage[english]{babel}
\usepackage[ansinew]{inputenc}
\usepackage[orientation=portrait,size=a0,scale=1.25,debug]{beamerposter}
\usepackage{booktabs,array}
\usepackage{listings}
%\usepackage{picins,graphicx}
\usepackage{xspace}
\usepackage{fp}
\usepackage{ifthen}
\usepackage[T1]{fontenc}
\usepackage{graphicx}
\usepackage{tikz,colortbl,pgf,pgfarrows,pgfnodes,pgfautomata,pgfheaps,pgfshade,eurosym, dsfont}
\listfiles
\newcommand*{\signstream}{SignStream\texttrademark\xspace}
\newcommand{\WichtigFarbe}{\color{red}}%
\newcommand{\TextFarbe}{\color{black}}%
\newcommand{\Pheight}{\rule[-5mm]{0cm}{1cm}}

\definecolor{darkblue}{rgb}{0.28,0,0.60}
\definecolor{hblue}{rgb}{0.70,0.7,1}
\definecolor{NR0}{rgb}{1,1,1}
\definecolor{NR1}{rgb}{1,1,0.8}
\definecolor{NR2}{rgb}{1,1,0.5}
\definecolor{NR3}{rgb}{1,1,0.25}
\definecolor{NR4}{rgb}{1,1,0.0}
\definecolor{NR5}{rgb}{1,0.75,0.0}
\definecolor{NR6}{rgb}{1,0.5,0.0}
\definecolor{NR7}{rgb}{1,0.25,0.0}
\definecolor{NR8}{rgb}{1,0,0.0}

\setbeamertemplate{navigation symbols}{}
\setbeamerfont{title}{series=\bfseries}
%\setbeamercolor{frametitle}{fg=UTblue}
\setbeamerfont{frametitle}{series=\bfseries}
\setbeamertemplate{frametitle}
{
\begin{centering}
\insertframetitle\vspace*{-4mm}\par
\end{centering}
}

\newcommand{\convD}{\stackrel{\text{d}}{\longrightarrow}}
\newcommand{\E}{\text{E}\,}
\newcommand{\var}{\text{V}\,}

\title{\huge Hier steht der Titel }
\author{\large Autor 1 und Autor 2}
\institute{Universit�t Bamberg} % (oder halt Berlin)

\date{today}
\begin{document}

\begin{frame}

%
%------------------------------------------------------ ------------------------------------------------------
%
\small
\begin{columns}[t] % Wechsel in die Spaltenumgebung

\begin{column}{.3\linewidth} 
\begin{block}{Erkl�rung Vorgehen  \Pheight}
\begin{enumerate} % f�r Stichpunkte
\item Das Thema ist besonders relevant, weil ...
\item Au�erdem ...
\item Weiterhin ...

\end{enumerate}

\end{block}
\end{column}

%
%------------------------------------------------------ ------------------------------------------------------
%

\begin{column}{.3\linewidth}
\begin{block}{�bersicht Genetic Matching Algorithmus\Pheight}


\end{block}
\end{column}

%
%------------------------------------------------------ ------------------------------------------------------
%

\begin{column}{.3\linewidth}
 \begin{block}{Genetic Matching Theorie \Pheight}

\begin{itemize}
\item Genetic Matching ist ein verallgemeinertes Distanzma� (Gleichgewichtung der Kovariaten entspricht der Mahalanobis Distanz)
\item Das Verfahren verwendet \textbf{GENOUD} (\textbf{Gen}etic \textbf{O}ptimization \textbf{U}sing \textbf{D}erivates)
\begin{itemize}
\item Basierend auf neun heuristischen Regeln werden neue Generationen an Gewichtungen erzeugt
\item durchschnittlich werden die angelegten Kriterien so von Generation zu Generation verbessert
\end{itemize}
\item Genetic Matching ist mit anderen Matching Methoden verkn�pfbar (z.B. Nearest Neighbour)
\item �ber eine Verlustfunktion (paarweise t-Tests, KS-Tests) wird die beste Gewichtung der Kovariaten zur Erstellung der n�chsten Generation ausgew�hlt
\item In jedem Schritt wird der kleinste p-Wert der standardisierten Tests maximiert

\end{itemize}


\end{block}
\end{column}

%
%------------------------------------------------------ ------------------------------------------------------
%

\end{columns}


\begin{columns}[t]


%
%------------------------------------------------------ ------------------------------------------------------
%


  
\begin{column}{.3\linewidth}
    
    


    \begin{block}{Theorie \Pheight}

Formeln macht man so:
$$\int_{a}^{b} f(x)\, dx \approx (b-a)\frac{f(a) + f(b)}{2}$$


\end{block}
\end{column}


%
%------------------------------------------------------ ------------------------------------------------------
% Folie 5
   
\begin{column}{.3\linewidth}

 \begin{block}{Umsetzung \Pheight}

Hier wird das Vorgehen erkl�rt:
\begin{itemize}
\item ...
\item ...
\item ...
\end{itemize}
\end{block}
\end{column}     



%
%------------------------------------------------------ ------------------------------------------------------
% Folie 6 

  % 


\begin{column}{.3\linewidth}


\begin{block}{Umsetzung \Pheight}
Ein Algorithmus zur L�sung des Problems:
\begin{enumerate} % F�r Aufz�hlungen
\item W�hle Startwerte f�r die Parameter.
\item F�lle die fehlenden Daten auf.
\item Berechne �ber die aufgef�llten Daten neue Parameterwerte.
\item F�hre Schritte 2 und 3 bis zur Konvergenz aus.
\end{enumerate}
\end{block}



\end{column}



%
%------------------------------------------------------ ------------------------------------------------------
%
  
\end{columns}  



\begin{columns}[t]

%
%------------------------------------------------------ ------------------------------------------------------
% 10 


\begin{column}{.3\linewidth}

\begin{block}{(Simulations-) Ergebnisse \Pheight}
So schreibt man \textbf{fett}.
\end{block}

\end{column}

%
%------------------------------------------------------ ------------------------------------------------------
% 11

\begin{column}{.3\linewidth}

\begin{block}{(Simulations-) Ergebnisse \Pheight}
\textit{Kursiv} geht auch
\end{block}

\end{column}

%
%-------------------------------------------------------------------------------------------------------------
% 12

\begin{column}{.3\linewidth}


  \begin{block}{ Fazit \Pheight}
\begin{itemize}
\item Erkennisse 
\item Schlussfolgerungen
\item Ausblick.
\item 
\end{itemize}
   \end{block}

\begin{tiny}
 \begin{thebibliography}{sotief}

    \bibitem{Muennich} Burgard, J.P.; M\"{u}nnich, R. (2010): Modelling over and undercounts for design-based Monte Carlo studies in small area estimation: An application to the German register-assisted census. Computational Statistics and Data Analysis.
    
    \bibitem{Muennichb} Gabler, S,; Ganninger,M.; M\"{u}nnich, R. (2010): Optimal allocation of the sample size to strata under box constraints. Metrika. 
    
    \bibitem{Gelman} Gelman, A.; (2007): Struggles with Survey Weighting and Regression Modeling. Statistical Science.
    \bibitem{Alfons} Alfons, A.; Filzmoser, P.; Hulliger, B.,Kolb, J.P.; Kraft, S.; M\"{u}nnich, R. und Templ, M. (2011):  The AMELI simulation study. Research Project Report WP6 - D6.1, FP7-SSH-2007-217322 AMELI. 
    \bibitem{Alfons} Alfons, A.; Filzmoser, P.; Hulliger, B., Kolb, J.-P.; Kraft, S.; M\"{u}nnich, R., und Templ, M. (2011): Synthetic data generation of SILC Data. Research Project Report WP6 - D6.2, FP7-SSH-2007-217322 AMELI. 
    
  \end{thebibliography}
\end{tiny}



\end{column}

%
%------------------------------------------------------ ------------------------------------------------------
%

\end{columns}

\end{frame}


\end{document}

